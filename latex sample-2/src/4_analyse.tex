\section{Analyse und Aufgabenstellung}\label{analysis}

Open Badges können über das criteria und alignment field mit Kompetenzen aus externen Kompetenzframeworks verlinkt werden. Daraus ergibt sich die Anforderung Badges mit ähnlichen Kompetezen, Badges die sich ergänzen oder identisch sind zu ermitteln, aber auch fehlende Kompetenzen zu identifizieren die benötigt werden um einen gewünschten Badge zu erlangen. \cite{OBNO3-A2}

\vspace{1em}

Diese Arbeit soll Methoden und Metriken evaluieren um Kompetenzen in einem Graphen zu modellieren und Ähnlichkeiten festzustellen.  Die Kompetenzen werden einem Kompetenzverzeichnis entnommen welches Kompetenzen aus verschiedenen Kompetenzrahmen verwaltet. Das Ergebnis wird ein Prototyp eines Empfehlungsystems sein, welches folgende Anforderungen erfüllen sollte.

\subsection{Anforderungsanalyse} 

Ein Empfehlungssystem bekommt eine Kompetenz übergeben und Liefert eine oder mehrere Kompetenzen zurück welche in Beziehung zur gegebenen Kompetenz stehen.

\subsubsection{Anforderungen an die Engine}

\begin{itemize}
	\item Ähnliche Kompetenzen finden
	\item Identische Kompetenzen finden
	\item Wichtigkeit von Kompetenzen anzeigen
\end{itemize}

Optionale Anforderungen


\subsubsection{Anforderungen an die Datenbank}

\subsection{Aufgabenstellung}

