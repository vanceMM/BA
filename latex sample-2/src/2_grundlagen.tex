\section{Grundlagen und Begrifflichkeiten}\label{2_grundlagen}

%Kompetenzen
%RDF Semantic Web & Linked Data
%Graphen
%Graphentheorie 
%Graphdatenbanken

\subsection{Kompetenzen}\label{competencies}

Im Allgemeinen spricht man bei der Verbindung von Wissen und Können um geforderte Handlungen zu bewältigen von Kompetenzen. Dabei liegt der Fokus auf Fähigkeiten und Wissen, welche dazu beitragen Probleme mit nicht standardmäßigen Handeln zu lösen, und auf verschiedene Situationen zu übertragen.\cite{bibb} Oftmals werden die beiden Begriffe Kompetenz und Qualifikation als Synonyme verwendet, jedoch gibt es signifikante Unterschiede in deren Bedeutung.
\vspace{1em}

Kompetenz (lateinisch: competere, zu etwas fähig sein) meint Lernerfolg im Hinblick auf den Lernenden selbst und seine Befähigung zu selbstverantwortlichem Handeln im privaten, beruflichen und gesellschaftlichen Bereich. Sie bezeichnet die subjektive Leistungsfähigkeit einer Person, welche nicht überprüfbar und objektiv Bewertbar ist.\cite{Bauer2015}\newline

Qualifikationen (lateinisch: qualis facere, Beschaffenheit herstellen) sind hingegen prüfbar und Zertifizierbar. Sie sind die äußere Seite der Leistungsanforderung und auf die Erfüllung vorgegebener Zwecke gerichtet. \cite{Bauer2015}


\subsection{Kompetenzframeworks}

Kompetenzen abzubilden gestaltet sich schwierig, da diese einen dynamischen Charakter hat um vom Kontext, ihrer Domäne abhängig ist. Durch die Unterteilung in Branchen und Sektoren ist eine Klassifizierung von Kompetenzen dann möglich. Eine weitere Dimension zur Klassifikation ist das Niveau, es beschreibt das Maß oder auch den \emph{Schwierigkeitsgrad} einer Handlung die mit der gegebenen Kompetenz zu bewältigen ist. 
\vspace{1em}

Da Abschlüsse keine Kompetenz bescheinigen, werden Qualifikationsrahmen benötigt, mit deren Hilfe die in einem Bildungsweg erworbenen Kenntnisse und Fähigkeiten zu Kompetenzentwicklung beitragen können. Mit diesen Lernergebnissen die das \emph{Können} im Sinne von \emph{in der Lage sein, etwas zu tun} beschreiben, ergeben sich dann Vergleichsmöglichkeiten.
\vspace{1em}

Um Qualifikationen und Kompetenzen aus unterschiedlichen sektoralen oder nationalen Rahmenwerken zu vergleichen benötigt es einen Meta-Rahmen. 
\vspace{6em}

Zusammenfassung:

\begin{quote}
	"Kompetenzrahmen sind eine wichtige Brücke zwischen dynamischer Kompetenz und statischen Bildungs- und Zertifikatssystemen.
Denn Kompetenzrahmen machen Kompetenz Klassifizierbar und unterstützen ihre Anerkennung und Anrechnung."
\end{quote}
\todo{zitieren?}


\subsection{RDF, Semantic Web  Linked Data}\label{semantic_web}
\subsubsection{Semantic Web}
Das Semantic Web stellt Daten, im World Wide Web in einem für Maschinen verarbeitbaren Art und Weise zur Verfügung mit dem Ziel der Interoperabilität, also der Möglichkeit Informationen zwischen Anwendungen und Plattformen auszutauschen und diese in Beziehung zu setzen. Kernaspekte sind das Auffinden relevanter Informationen, die Integration von Informationen aus verschiedenen Quellen und automatische Schlussfolgerung. Die Ziele des Semantik Web lassen sich grob zusammen fassen als Absicht Methoden und Wege zu finden um "Informationen so zu repräsentieren, dass Maschinen damit in einer Art und Weise umgehen können, die aus menschlicher Sicht nützlich und sinnvoll erscheint.”\cite[12]{Hitzler2007}\newline
 
Im folgenden sollen einige Konzepte und Technologien erläutert werden welche die Anforderungen des Semantic Webs umsetzen.
 
\subsubsection{Linked Data}

Die Beziehungen der Daten im Semantic Web müssen nicht auf ein Datensatz beschränkt bleiben. So können Daten auch Querverweise auf andere Datensätze haben. Ein Beispiel sind die Daten von Wikipedia die durch das Linked Dataset DBPedia zugänglich sind. Einzelne Einträge beinhalten dabei Verweise auf das Geonames Dataset.
Einige dieser Datensätze sind öffentlich zugänglich (Linked Open Data). So stellt das Amt für Veröffentlichungen der Europäischen Union mit ihrem offenen Datenportal Daten ihrer Institutionen und anderer Einrichtungen zur Verfügung. \newline
 
Die Vorgehensweise bei der Erstellung von Linked Data wird von Tim Berners-Lee\cite{ld} in folgenden 4 Schritten beschrieben:

\begin{enumerate}
	\item Dinge und Objekte werden durch URIs identifiziert
	\item Die URIs sind über das HTTP Protokoll aufrufbar
	\item Beim Aufruf werden relevante Informationen in standardisierten Formaten geliefert
	\item Die gelieferten Daten enthalten Referenzen auf andere URIs
\end{enumerate}
\subsubsection{RDF}

Um die Anforderung des Semantic Web’s, Daten im Web auszutauschen, zu erfüllen, wurde das Resource Description Framework RDF als eine formale Sprache für die Beschreibung strukturierter Informationen geschaffen. Dabei soll die ursprüngliche Bedeutung erhalten bleiben und Kombinationen und Weiterbearbeitung der enthaltenen Informationen ermöglicht werden.
Eine Resource kann generell jedes Objekt mit einer eindeutigen Identiät sein, wie zb. Bücher, Orte, Menschen oder abstrakte Konzepte. Um Mehrdeutigkeiten zu verhindern werden URIs als Bezeichner verwendet. Diese RDF-Beschreibungen können auch durch Zeichenketten syntaktisch dargestellt werden, müssen vorher jedoch in Bestandteile zerlegt und serialisiert werden. Die dabei entstehenden Dokumente sind gerichtete Graphen, wobei Knoten und Kanten mit eindeutigen Bezeichnern beschriftet sind. \newline
 
Triple RDF-Graphen lassen sich vollständig durch ihre Kanten beschreiben. Eine solche Kante hat einen Anfangspunkt, eine Beschriftung und einen Endpunkt. Dieses Triple wird bestimmt durch \emph{Subjekt-Prädikat-Objekt}.

\subsubsection{Ontologie/Vokabular}

Als Ontologie oder Vokabular wird eine Sammlung von Begriffen bezeichnet, die innerhalb einer Domäne \textit{Wissen} abbilden und klassifiziert werden. \cite{w3c} Mit Hilfe eines Vokabulars werden Regeln über die Semantik von Klassen, Attributen und Beziehungen von Daten definiert. Ontologien können auch in einer spezifischeren Domäne erweitert werden. In der \textit{Simple Knowledge Organization System} Ontologie wird durch die \textit{Concept} Klasse eine Idee oder ein Sachverhalt abgebildet und mit anderen verknüpft. Auch Vererbung und hierarchische Strukturen sind möglich.


\todo{Ontologienen ausarbeiten}

\subsection{Graphen}

Ein Graph ist definiert als eine Menge von Knoten (Vertices) und deren Beziehungen, welche über Kanten (Edges) dargestellt werden. Mit Graphen kann man vernetzte Strukturen wie zb. Straßennetze, Computernetzwerke oder Datenstrukturen modellieren. So finden sich in vielen modernen Technologien wie Routenplaner oder Social Media Anwendungen graphentheoretische Konzepte wieder.
	\subsubsection{Graphentheorie}
	Die Graphentheorie is ein Teilgebiet der Mathematik, in dem Graphen und ihre Beziehungen zueinander untersucht. Das älteste dokumentierte Problem der Graphentheorie ist das Königsberger Brückenproblem. Dabei wurde ein Rundweg durch Königsberg gesucht, der alle Brücken jedoch nur einmal überquerte. Leonhard Euler erkannte 1736, dass man die einzelnen Ufer als Punkte und Brücken als Kanten abstrahieren konnte. Euler zeigte, dass ein solcher Weg nicht existierte, da jeder Knoten mit einer ungeraden Anzahl von ungerichteten Kanten verbunden sein muss.
	\newline
	
	Da dieses Themengebiet sehr umfassend ist sollen hier nur einige für diese Arbeit relevante Konzepte erwähnt werden.\newline
	
	\textbf{Gerichteter Graph: }
	
	Ein gerichteter Graph ist definiert als G = (V,R, $\alpha$ , $\omega$) mit V als nicht leere Menge von Knoten, R als Kantenmenge.  $\alpha$ und $\omega$ sind jeweils Abbildungen für die gilt r($\alpha$,$\omega$), wobei r eine Kante ist die ihren Ursprung im Anfangsknoten $\alpha$ hat und im Endknoten $\omega$ endet. \cite[7]{Krumke2012}
	\newline
	
	\textbf{Adjazenmatrix:}
	
	Die Adjazenzmatrix oder auch Nachbarschaftsmatrix ist eine  $n\times n $Matrix mit n = |V|. Sie gibt welche Knoten im Graphen miteinander verbunden sind. Für die Adjazenzmatrix \textit{A(G)} gilt $a_{aj}$ = | \{ $r \in R: \alpha(r) = v_{i}$ und $\omega(r) = v_{j} $ \} |. \cite[17ff]{Krumke2012}\newline	

	\textbf{Gewichtete Kanten:}
	
	Eine Kante kann bewertet werden. Sie wird mit einer Zahl notiert. Diese Zahl kann dann als Parameter verwendet werden und gibt die Kosten an, die anfallen wenn die Kante von einem Algorithmus oder einer Funktion passiert wird.
	\cite[18]{Krumke2012}
	
	
	
	\subsubsection{Graphendatenbanken}	
	Im Gegensatz zu relationalen Datenbanken werden Daten nicht in Tupeln als Tabellen gespeichert sondern als Knoten in einem Graphenmodell. Der Vorteil darin liegt zunächst in der Daten Modelierung, da das Graphen Model sehr intuitive ist und sich einfach auf Papier modelieren lässt. 
	Ein weiterer wichtiger Aspekt ist der Performance Vorteil bei Abfragen von stark verknüpften Daten. Abfragen in relationalen Datenbanken werden langsamer, je größer der Datenbestand. Im Graphen wird immer nur der Teilgraph durchlaufen welcher, die die Abfrage erfüllt.\cite[8]{robinson_webber_eifrem_2015}
	
	\subsubsection{Neo4J}
	
	Die Open Source Graphdatenbank Neo4j ist durch ein \textit{Property Graph Model} implementiert. Knoten im Graph können durch \emph{Label (Beschriftungen)} in Kategorien eingeordnet werden und speichern Informationen in Schlüssel-Wert Paaren als \emph{Properties (Eingeschaften)}. 
	Neben Knoten können auch Kanten Eigenschaften erhalten, was eine Gewichtung der Beziehungen zwischen Knoten ermöglicht, indem man z.B. eine Zahl als Eingeschaft für eine Kante speichert. Eine Beziehung besteht immer aus einem Start- und Endknoten, sowie einer Richtung.\cite[26]{Robinson2015}
	Anfragen an die Neo4j Datenbank werden mittels der deklarativen Sprache \textit{Cypher} ausgeführt. Mit dieser Sprache können Daten gefunden werden, welche einem Muster  entsprechen. Diese Patterns setzen sich aus Knoten und Beziehungen zusammen.\newline
	
%	\begin{figure}[htb]
% \centering
% \includesvg{abb/cypher}
% \caption[Beschreibung]{Cypher mit der Beziehung "mag"}
%\label{fig:Beschreibung}
%\end{figure}

Mit der \textit{MATCH} Klausel, werden alle Pfade im Graphen gezeigt, welche dem Pattern entsprechen. Mit \textit{WHERE} kann die Sammlung gefiltert werden.

\subsection{Empfehlungssysteme}

Ein Empfehlungsystem ist eine Software, welche Nutzern Vorschläge zu Artikeln, Music, Filmen, Büchern oder anderen Objekten macht. \cite{Ricci2010} Ausgehend von einem Objekt werden dem Benutzer andere Objekte mit Ähnlichkeiten zu dem Ausgangsobjekt aufgelistet. Für die Ergebnisse werden im wesentlichen zwei Ansätze verfolgt. Das Inhalts basierte Filtern und Kollaborative Filtern. 
Beim Inhaltsbasierenden Filtern werden Objekte mit einander verglichen. Ein einfacher Anwendungsfall sind zb. Produkt Empfehlungen basierend auf Eigenschaften der Produkte. Wählt ein Benutzer ein Produkt aus, so werden ihm Produkte mit ähnlichen Eigenschaften aufgelistet. 
Mit dem Kollaborativen Filtern wird das Verhalten von Nutzern verglichen. Ein Empfehlungssystem für Online Shops, kann dahingehend erweitert werden, dass nach Produkten gesucht wird die andere Benutzer ebenfalls gekauft haben. 

	