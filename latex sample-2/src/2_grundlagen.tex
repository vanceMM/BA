\section{Grundlagen}\label{2_grundlagen}

%Kompetenzen
%RDF Semantic Web & Linked Data
%Graphen
%Graphentheorie 
%Graphdatenbanken

\subsection{Kompetenzen}\label{competencies}

Im Allgemeinen spricht man bei der Verbindung von Wissen und Können um geforderte Handlungen zu bewältigen von Kompetenzen. Dabei liegt der Fokus auf Fähigkeiten und Wissen welche dazu beitragen Probleme mit nicht standardmäßigen Handeln zu lösen, und auf verschiedene Situationen zu übertragen.\cite{bibb}

Oftmals werden die beiden Begriffe Kompetenz und Qualifikation als Synonyme verwendet, jedoch gibt es signifikante Unterschiede in deren Bedeutung. 
\vspace{1em}
 
Kompetenz(lateinisch: competere, zu etwas fähig sein) meint Lernerfolg im Hinblick auf den Lernenden selbst und seine Befähigung zu selbstversantwortlichem Handeln im privaten, beruflichen und gesellschaftlichen Bereich. Sie bezeichnet die subjektive Leistungsfähigkeit einer Person, welche nicht überprüfbar und objektiv bewertbar ist.

Qualifikationen(lateinisch: qualis facere, Beschaffenheit herstellen) sind hingegen prüfbar und zertifizierbar. Sie sind die äußere Seite der Leistungsanforderung und auf die Erfüllung vorgegebener Zwecke gerichtet. 
 
Fachkompetenzen
Beispiele für Kompetenzen und Qualifikationen:

\subsection{Kompetenzframeworks}
\subsection{RDF, Semantic Web  Linked Data}\label{semantic_web}
Das Semantic Web stellt Daten im World Wide Web in einem für Maschinen verarbeitbaren Art und Weise zur Verfügung mit dem Ziel der Interoperabilität, also der Möglichkeit Informationen zwischen Anwendungen und Plattformen auszutauschen und diese in Beziehung zu setzen. Kernaspekte sind das Auffinden relevanter Informationen, die Integration von Informationen aus verschiedenen Quellen und automatische Schlussfolgerung.
“Finde Wege und Methoden, Informationen so zu repräsentieren, dass Maschinen damit in einer Art und Weise umgehen können, die aus menschlicher Sicht nützlich und sinnvoll erscheint.”
 
Im folgenden sollen einige Konzepte und Technologien erläutert werden welche die Anforderungen des Semantic Webs umsetzen.
 
		
\cite[12]{Hitzler2007}


\subsection{Graphen}

Ein Graph(G) ist eine Struktur, welche aus Knoten(V) besteht die wiederum durch Kanten(E) verbunden sein können.

	\subsubsection{Graphentheorie}
	\subsubsection{Graphendatenbank}
		