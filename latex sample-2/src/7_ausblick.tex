\section{Ausblick}\label{ausblick}

Der Prototyp des Empfehlungsdienstes ist in seinen Grundzügen fertig implementiert, kann und soll erweitert und verbessert werden. So können weitere Daten mit Kompetenzen in die Datenbank geladen werden, oder neue Beziehungen eingeführt. Eine wichtige Eigenschaft des Neo4j \textit{Label Property Graph} Models, die Gewichtung von Kanten, wird bisher noch nicht genutzt. Der Europäische Qualifikationsrahmen sieht aber Niveau Levels vor, mit diesen könnte sich die Kosinus Ähnlichkeit als weitere Metrik berechnen lassen. 
Die implementierten Methoden zur Ähnlichkeitsmessung sind lediglich ein Ansatz, für einen Vollfunktionsumfänglichen Empfehlungsdienst müssten die einzelnen Metriken evaluiert und gegebenenfalls angepasst werden. Im Idealfall werden diese nicht nur einzeln angewandt, sondern mit einander kombiniert. \newline

  Zum Zeitpunkt der Entscheidung die Daten von ESCO zu verwenden lagen standen diese nur in einer frühen Version 0.8 zur Verfügung. Seit dem 28.07.2017 sind die Daten nun in überarbeiteter Struktur verfügbar und es wurden z.B. für die Informations- und Technologiebranche relevante Sektor spezifische Kompetenzen hinzugefügt. 
  \newline
 Die Europäische Kommission ist momentan die einzige Organisation, welche Kompetenzen auf Basis eines Standards(EQF) als maschinenlesbares Format veröffentlicht. Wenn andere Organisation in Zukunft diesem Beispiel folgen, stellt sich die Frage ob ein einheitliches Format notwendig ist damit ein Empfehlungssystem, wie das in dieser Arbeit entwickelte, damit umgehen kann. In diesem Fall könnte der Lösungsansatz des inLOC Projektes verwendet werden. 