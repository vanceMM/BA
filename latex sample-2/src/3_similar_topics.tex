\section{Verwandte Themen und Arbeiten}\label{similar_topics}

\subsection{Openbadge}\label{openbadge}

Auf unserem Bildungsweg werden das Erlangen von Fertigkeiten und Kenntnissen mit Zeugnissen und Abschlüssen belegt. Doch oftmals genügt die formale Ausbildung nicht oder hat aufgrund der sich schnell verändernden Technologien oder Kompetenzen nur eine begrenzte Gültigkeit.  
Die Europäische Union fordert eine stärkere Anerkennung von informalem Lernen, damit auch Fertigkeiten und Kenntnisse die ohne ein formales Abschlusszertifikat erworben wurden Anerkennung finden.\cite{Dorn2014}

Doch wie können Personen alle Ihre Fertigkeiten, welche an einer Hochschule, in einer staatlich Anerkannten Ausbildung oder  in einem Online Seminar, in einem Workshop etc. erworben wurden präsentieren, damit auch Arbeitgeber und Bildungsinstitute in der Lage sind, sicherzustellen, dass Bewerber die nötigen Fertigkeiten mitbringen.\cite{alliance_for_excellent_education}

In der digitalen Welt können sogenannte Badges ein Lösungsansatz sein. Ein digitaler Badge ist ein digitales Zertifikat für eine erbrachte Leistung oder eine Fähigkeit. 
Die Mozilla Foundation hat in Zusammenarbeit mit der MacArthur Foundation den Open Badge(OB) Standard entwickelt. Er stellt sicher, dass Alle Badges Informationen über Kriterien und Nachweise erhalten. Die Informationen in einem Badge können auch auf ein Kompetenzframework verweisen, und validiert werden.\cite[4]{alliance_for_excellent_education}

\vspace{1em}

Badges können von Institutionen, Schulen und Arbeitgebern verliehen werden. Sie definieren ein Set von Kompetenzen oder einen Lehrplan und eine Bewertung um festzustellen ob ein Empfänger die notwendigen Anforderungen erfüllt hat. Darüber hinaus können Badges von ihrem "issuer" mit einer verschlüsselten Zusicherung versehen werden, welche bestätigt, dass der "earner" des Badges die geforderte Leistung auch erbracht hat. 
Die Zusicherung kann dann in den Quellcode eines SVG oder PNG Bildes geschrieben werden, sodass dritte später eine elektronisch Überprüfung beim Herausgeber beantragen können. Über das Alignment-Attribut kann ein Badge auch auf eine Quelle verweisen, welche die Kompetenz oder Fähigkeit beschreibt.

\vspace{1em}
\subsection{Europäischer Qualifikationsrahmen}
Der EQR ist ein Meta-Rahmen, welcher die Vergleichbarkeit von beruflichen Qualifikationen und Kompetenzen aus verschiedenen nationalen oder sektoralen Kompetenzrahmen ermöglichen soll. \todo{ausführung, nutzung von eqf in ESCO}
vergleichbarkeit von beruflichen Qualifikationen und Kompetenzen
\subsection{European e-Competence Framework}\label{e-CF}

Der europäische Kompetenzrahmen für Fach- und Führungskräfte der Informations- und Kommunikationstechnologiebranche, ist eine sektor-spezifische Umsetzung des Europäischen Qualifikationsrahmens EQR. Er unterteilt Kompetenzen in 5 Feldern auf 5 Niveaus. 
\subsection{InLoc}\label{inloc}

Das Europäische Projekt InLOC(Integrating Learning Outcomes and Competences) erlaubt es Kompetenzen und Lernergebnisse verschiedener Kompetenzrahmen in einem einheitlichen semantischen Format abzubilden. 


\todo{Wichtig? Mehr Informationen ausarbeiten}

\subsection{ESCO}

In der EU gibt es nach einer Aktuellen Eurostat Statistik ca. 19 Millionen Menschen ohne Beschäftigung. Jedoch haben einige Branchen in Deutschland Probleme Stellen mit qualifiziertem Personal zu besetzen. So blieben im Jahr 2016 In der IT und Telekommunikationsbranche 375.034 Stellen unbesetzt.\cite{Statista2016}
 
Die Europäische Kommission hat dieses Problem erkannt und mit ESCO eine mehrsprachige Klassifizierung für europäische Fähigkeiten, Kompetenzen, Qualifikationen und Berufe entwickelt, deren Zusammenhang durch Berufsprofile verdeutlicht wird.
 
Ein der Aufgaben von ESCO ist es, Die Lücken zwischen dem Arbeitsmarkt und den verschiedenen Bildungssystemen der einzelnen Mitgliedstaaten zu schließen. So unterscheiden sich Qualifikationen welche Menschen in ihren Heimatländern erhalten nicht nur voneinander, sondern können auch oftmals nicht mit aktuellen Entwicklungen des Arbeitsmarktes und dessen Anforderungen mithalten.

Die ESCO Daten werden gemäß den Praktiken für Linked Open Data veröffentlicht. Dies soll Entwicklern den Zugriff erleichtern und Anwendungen für Stellenausgleich, Berufsberatung und Selbsteinschätzungen ermöglichen.

Die Europäische Kommission hat mit ESCO eine Schnittstelle geschaffen, welche Informationen zwischen den nationalen Klassifizierung Systemen übersetzen soll und somit eine höhere semantische Interoperabilität schaffen wird. Ein Hauptinteresse von ESCO ist der kompetenzbasierte Job Abgleich. Arbeitnehmer sollen ihre eigenen Fähigkeiten, Kompetenzen und Qualifikationen mit freien Stellen vergleichen können, und so eventuelle Kompetenzlücken identifizieren zu können. Auf der anderen Seite, muss es Arbeitgebern möglich sein, Stellenausschreibungen durch  Fähigkeiten, Kompetenzen und Qualifikationen zu beschreiben, und Bewerber mit den geforderten Kompetenzen abzugleichen. Diese Anforderungen müssen nun von IT-System erfüllt werden. 

\subsubsection{EURES}


\subsubsection{GraphGist: Recommendation System Sandbox}\label{recommender}

Neo4J bietet über die Sandbox eine interaktive Möglichkeit auf einer temporär generierten Instanz im Browser zu arbeiten. Mit Schritt-für-Schritt Anleitungen werden Themen wie "Netzwerk Management" oder die "Panama Papers" in Neo4j näher gebracht. 

Ein für diese Arbeit relevantes Themenfeld bietet die Sandbox "Recommendations".\cite{neo4j} Als Datenquelle für die Instanz stehen die "Open Movie Database"\cite{omdb}, und das MovieLens Projekt\cite{grouplens_2016} zur Verfügung.
Neben einer Erläuterung zum Property Graph Model und einer Einführung in die Cypher Query Language gibt es Beispiele zu verschiedenen Methoden und Metriken für das Filtern von Ergebnissen. Dabei werden verschiedene Ansätze zu den Methoden Inhaltsbasierte Filterung und Kollaborative Filterung erläutert. 
\subsection{RDF Import Neo4J}

Jesús Barrasa ist  \textit{Senior Graph Solutions Consultant at Neo Technology} bei Neo4j, und hat ein Plugin für Neo4j entworfen, mit dessen Hilfe sich RDF Dokumente in Neo4j importieren lassen. So lässt sich mit dem Befehl :
\vspace{1em}

\begin{lstlisting}[frame=htrbl, caption={Das Listing zeigt einen Funktionsaufruf über die Neo4j}, label={lst:result2}]
CALL semantics.importRDF("file:////..../esco_skos.rdf","RDF/XML",
 { languageFilter: 'de', commitSize: 5000 , nodeCacheSize: 250000})	
\end{lstlisting}

Der komplette RDF Graph des ESCO Kataloges mit allen Beziehungen laden und Abfragen.

%\begin{figure}[htb]
% \centering
% \includegraphics[width=0.3\textwidth,angle=0]{abb/rdf_import_neo4j}
% \caption[Beschreibung]{Beschreibung}
%\label{fig:Beschreibung}
%\end{figure}

\subsection{Recommender Systeme}

Bekannte Anwendungsfälle für ein Recommender System sind neben Musik und Film Empfehlungen vor allem die \textit{andere Kunden kauften auch...} Komponente in Web Shops. Ziel dieser Systeme ist es eine Vorhersage zu treffen, welches Produkt oder Objekt dem Kunden oder Anwender ebenfalls interessieren könnte. Dabei werden häufig die Konzepte des \textit{Inhaltsbasierenden} und des \textit{Kollaborativen} Filtern angewandt. 

Beim Inhaltsbasierten Filtern wird die Ähnlichkeit der Objekte und deren Eigenschaften ermittelt. Im Gegensatz dazu werden beim kollaborativen Filtern Benutzer und deren Präferenzen miteinander verglichen. 


\subsubsection{GraphGist Recommendation Engine Neo4j}

\subsubsection{RecSys Challenge}

Die \textit{ACM RecSys Conference} befasst sich jedes Jahr mit den aktuellen Untersuchungsergebnissen und Techniken im Bereich der Empfehlungsdienste. Darüber hinaus veranstaltet sie auch die RecSys Challenge, einen mit 3000 \euro  Siegerprämie dotierten Wettbewerb, bei dem Teilnehmer ein Empfehlungsdienst, in einer bestimmten Domäne entwickeln sollen. In den beiden vergangen Jahren beschäftigte sich der Wettbewerb mit Job Empfehlungen der Karriereplatform Xing. Die Aufgabe war, anhand eines neuen Job Angebotes, all jene Nutzer zu finden, die interessiert sein könnten eine Benachrichtigung über das neue Angebot zu erhalten aber auch Nutzer die ebenfalls für den Job geeignet sein könnten. Die Ergebnisse werden auf der RecSys Conference in Como, Italien am 27.08.2017 vorgestellt.
\subsubsection{Job-Matching: Xing}

Das Karrierenetzwerk bietet Arbeitssuchenden die Möglichkeit Job-Empfehlungen auf Basis des eigenen Profils zu erhalten. Dabei wird nach den Kriterien Entfernung zum Arbeitsort, Karrierestufe, Skills und Aktivitäten gefiltert. Aus diesen 4 Kriterien wird ein Relevanz-Indikator ermittelt, welcher die Treffgenauigkeit des Inserats angibt. Umgekehrt wird auch ein Inserat mit dem eigenen Profil gematcht, wobei auch die geforderten Skills den eigenen gegenübergestellt werden.\cite{hoelscher}

\begin{figure}[htb]
 \centering
 \includegraphics[width=0.7\textwidth,angle=0]{abb/xing_jobmatching}
 \caption[Beschreibung]{Jobmatching Xing-Profil}
\label{fig:Beschreibung}
\end{figure}

Dieser Ansatz berücksichtigt jedoch nicht die Persönlichkeit der Bewerber, und bietet auch nur eine teilweise höhere Bewerberpassgenauigkeit aus Sicht der Arbeitgeber. Wie qualifiziert ein Bewerber wirklich ist bleibt weiter unklar, da eine Liste der angegebenen Skills keine Auskunft darüber gibt wie kompetent der Bewerber auf einem Gebiet wirklich ist. So kann es passieren, dass ein Bewerber der trotz Eignung durch das Raster fällt, weil er andere Schlagworte als die geforderten angegeben hat.

Neben Xing bieten die Plattformen Truffles und BirdieMatch ähnliche Technologien an.




