\section{Verwandte Themen und Arbeiten}\label{similar_topics}

\subsection{Openbadge}\label{openbadge}

Auf unserem Bildungsweg werden das Erlangen von Fertigkeiten und Kenntnissen mit Zeugnissen und Abschlüssen belegt. Doch oftmals genügt die formale Ausbildung nicht oder hat aufgrund der sich schnell verändernden Technologien oder Kompetenzen nur eine begrenzte Gültigkeit.  
Die Europäische Union fordert eine stärkere Anerkennung von informalem Lernen, damit auch Fertigkeiten und Kenntnisse die ohne ein formales Abschlusszertifikat erworben wurden Anerkennung finden.\cite{Dorn2014}

Doch wie können Personen alle Ihre Fertigkeiten, welche an einer Hochschule, in einer staatlich Anerkannten Ausbildung oder  in einem Online Seminar, in einem Workshop etc. erworben wurden präsentieren, damit auch Arbeitgeber und Bildungsinstitute in der Lage sind, sicherzustellen, dass Bewerber die nötigen Fertigkeiten mitbringen.\cite{alliance_for_excellent_education}

In der digitalen Welt können sogenannte Badges ein Lösungsansatz sein. Ein digitaler Badge ist ein digitales Zertifikat für eine erbrachte Leistung oder eine Fähigkeit. 
Die Mozilla Foundation hat in Zusammenarbeit mit der MacArthur Foundation den Open Badge(OB) Standard entwickelt. Er stellt sicher, dass Alle Badges Informationen über Kriterien und Nachweise erhalten. Die Informationen in einem Badge können auch auf ein Kompetenzframework verweisen, und validiert werden.\cite[4]{alliance_for_excellent_education}

\vspace{1em}

Badges können von Institutionen, Schulen und Arbeitgebern verliehen werden. Sie definieren ein Set von Kompetenzen oder einen Lehrplan und eine Bewertung um festzustellen ob ein Empfänger die notwendigen Anforderungen erfüllt hat. Darüber hinaus können Badges von ihrem "issuer" mit einer verschlüsselten Zusicherung versehen werden, welche bestätigt, dass der "earner" des Badges die geforderte Leistung auch erbracht hat. 
Die Zusicherung kann dann in den Quellcode eines SVG oder PNG Bildes geschrieben werden, sodass dritte später eine elektronisch Überprüfung beim Herausgeber beantragen können. Über das Alignment-Attribut kann ein Badge auch auf eine Quelle verweisen, welche die Kompetenz oder Fähigkeit beschreibt.

\vspace{1em}

\subsubsection{InLoc}\label{inloc}
\subsection{ESCO}

In der EU gibt es nach einer Aktuellen Eurostat Statistik ca. 19 Millionen Menschen ohne Beschäftigung. Jedoch haben einige Branchen in Deutschland Probleme Stellen mit qualifiziertem Personal zu besetzen. So blieben im Jahr 2016 In der IT und Telekommunikationsbranche 375.034 Stellen unbesetzt. 
 
Die Europäische Kommission hat dieses Problem erkannt und mit ESCO eine mehrsprachige Klassifizierung für europäische Fähigkeiten, Kompetenzen, Qualifikationen und Berufe entwickelt, deren Zusammenhang durch Berufsprofile verdeutlicht wird.
 
Ein der Aufgaben von ESCO ist es, Die Lücken zwischen dem Arbeitsmarkt und den verschiedenen Bildungssystemen der einzelnen Mitgliedstaaten zu schließen. So unterscheiden sich Qualifikationen welche Menschen in ihren Heimatländern erhalten nicht nur voneinander, sondern können auch oftmals nicht mit aktuellen Entwicklungen des Arbeitsmarktes und dessen Anforderungen mithalten.
