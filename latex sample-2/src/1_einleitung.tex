\section{Einleitung}\label{einleitung}

Mit dem Abschluss einer jeden Berufsausbildung oder eines Studiums beginnt auch der Einstieg in das Berufsleben. Für Viele Bewerber bedeutet das, vermeintlich in Frage kommende Jobangebote zu sichten, und vergleichen. Es folgt das schreiben unzähliger individueller Bewerbungen von denen im schlimmsten Fall nur wenige zu einem Vorstellungsgespräch führen. Doch oftmals kommen schon bereits nach den ersten Wochen des Hochgefühls endlich eine Stelle zu haben, erste Zweifel ob der Job der richtige für einen ist, und die eigenen Wünsche und Anforderungen erfüllen kann.

Auch für Personaler stellt die Einstellung von neuem Personal eine schwierige Aufgabe dar. So müssen oft aus einer Reihe von Bewerbern die tatsächlich unqualifizierten aussortiert werden und eine Auswahl getroffen werden, welche Bewerber zu einem Gespräch eingeladen werden oder in die nächste Bewerbungsrunde gelangen.
 
Es besteht also Bedarf den Prozess von Arbeitsvermittlungen zu optimieren und zu vereinfachen. Das spart sowohl Bewerbern als auch Arbeitgebern viel Frust und Zeit. Ein automatischer Abgleich von Kompetenzen kann diesen Prozess beschleunigen, setzt aber einen einheitlichen Rahmen für die Beschreibung von Kompetenzen voraus. 
 \vspace{1em}
Erlangte Fähigkeiten und Wissen werden im allgemeinen durch Abschlüsse, Zertifikate oder Urkunden bestätigt. Jeder Arbeitnehmer hat einen oder mehrere Abschlüsse die ihn für einen bestimmten Beruf qualifizieren. Leider unterscheiden sich die Bezeichnungen dieser Abschlüsse, beinhalten aber oft dieselben oder ähnliche Kompetenzen. So gibt es neben dem normalen Informatikstudium auch Angebote wie: 
 

\begin{itemize}
  \item Wirtschaftsinformatik
  \item Angewandte Informatik
  \item Medizininformatik
\item Bioinformatik
\item Medieninformatik
\item Technische Informatik
\item Navigation und Umweltrobotik

\end{itemize}

 
Diese Studiengänge haben zwar unterschiedliche Inhalte, bauen aber auf denselben Grundlagen wie zb. Mathematik, Algorithmik, Datenbanken etc. auf. 