\section{Einleitung}\label{1_einleitung}

Mit dem Abschluss einer jeden Berufsausbildung oder eines Studiums beginnt auch der Einstieg in das Berufsleben. Für Viele Bewerber bedeutet das, in Frage kommende Jobangebote zu sichten, und vergleichen. Es folgt das schreiben unzähliger individueller Bewerbungen von denen im schlimmsten Fall nur wenige zu einem Vorstellungsgespräch führen. Doch oftmals kommen schon bereits nach den ersten Wochen des Hochgefühls endlich eine Stelle zu haben, erste Zweifel ob der Job der richtige für einen ist, und die eigenen Wünsche und Anforderungen erfüllen kann.

Auch für Personaler stellt die Einstellung von neuem Personal eine schwierige Aufgabe dar. So müssen oft aus einer Reihe von Bewerbern die tatsächlich unqualifizierten aussortiert werden und eine Auswahl getroffen werden, welche Bewerber zu einem Gespräch eingeladen werden oder in die nächste Bewerbungsrunde gelangen.
 
Es besteht also Bedarf den Prozess von Arbeitsvermittlungen zu optimieren und zu vereinfachen. Das spart sowohl Bewerbern als auch Arbeitgebern viel Frust und Zeit. Ein automatischer Abgleich von Kompetenzen kann diesen Prozess beschleunigen, setzt aber einen einheitlichen Rahmen für die Beschreibung von Kompetenzen voraus. 
% \vspace{1em}
%Erlangte Fähigkeiten und Wissen werden im allgemeinen durch Abschlüsse, Zertifikate oder Urkunden bestätigt. Jeder Arbeitnehmer hat einen oder mehrere Abschlüsse die ihn für einen bestimmten Beruf qualifizieren. Leider unterscheiden sich die Bezeichnungen dieser Abschlüsse, beinhalten aber oft dieselben oder ähnliche Kompetenzen. So gibt es neben dem normalen Informatikstudium auch Angebote wie: 
% 
%\begin{itemize}
%  \item Wirtschaftsinformatik
%  \item Angewandte Informatik
%  \item Medizininformatik
%\item Bioinformatik
%\item Medieninformatik
%\item Technische Informatik
%\item Navigation und Umweltrobotik
%
%\end{itemize}
%
%Diese Studiengänge haben zwar unterschiedliche Inhalte, bauen aber auf denselben Grundlagen wie zb. Mathematik, Algorithmik, Datenbanken etc. auf. 

Das Kernproblem bei der Arbeitsvermittlung besteht darin, eine vakante Stelle mit einem geeigneten Bewerber zu besetzen. Auf der einen Seite steht also der Bewerber der Kompetenzen anbietet, auf der anderen Seite der Arbeitnehmer, welcher gewisse Kompetenzen fordert. Das technische Problem welches hier umgesetzt werden muss ist, beide Seiten zusammen zu bringen. Der naive Ansatz wäre einfach die Liste der Kompetenzen des Stellenangebotes mit der Liste aus dem Lebenslauf des Bewerbers zu vergleichen und die Passgenauigkeit zu berechnen. Das führt aber unter Umständen dazu, dass interessante Bewerber aussortiert werden weil eventuell andere Bezeichnungen für ihre Fertigkeiten angegeben haben. 

Diese Problem wirft die Frage auf welche Standards und Richtlinien gibt es, um Kompetenzen abzubilden und zu modellieren 
und ob Quellen existieren um Ein Computer basiertes System zu entwickeln welches dies Kompetenzen mit einander vergleicht und Ähnlichkeiten findet. 

Für diese Vergleiche soll ein semantischer Ansatz gefunden werden, es sollen also keine Textvergleiche statt finden, sondern Kompetenzen sollen als Graph abgebildet sein.
Diese Arbeit soll sich mit Graphen Basierten Technologien und Ansätzen auseinander setzen, um Kompetenzen zu modellieren und Ähnlichkeiten zu finden.

\subsection{Inhalte dieser Arbeit}


Der Begriff der \textit{Kompetenz} wird in den Domänen Beruf und Bildung häufig verwendet, oft aber mit unterschiedlicher Definition und Bedeutung. So taucht der Begriff auch häufig in Verbindung mit anderen Substantiven auf. Beispiele sind: Fachkompetenz, Kernkompetenz, Handlungskompetenz, Teilkompetenz, Selbst- und Sozialkompetenz.
Um eine Abgrenzung zu schaffen soll zunächst der Rahmen für den Kompetenzbegriff in dieser Arbeit gespannt werden. 

Ein großes wissenschaftliches Interesse besteht im Bereich sogenannter \textit{Recommender Systeme}. Die \textit{Association for Computing Machinery} hält jährlich eine eigene Konferrenz zum Thema ab auf der, neuste wissenschaftliche Ergebnisse diskutiert werden. Ein solcher Empfehlungsdienst für die Domäne von Berufs Kompetenzen im Rahmen dieser Arbeit entworfen werden.

Die Arbeit soll die Möglichkeit evaluieren ob Kompetenzen als Graph modelliert werden können ,es also entweder eine hierarchische Baumstruktur gibt oder die Daten miteinander verknüpft sind. Darum soll eine Technologie ausgewählt um Kompetenzen in einer Graphen Struktur zu speichern, und mit Hilfe dieses Graphen Methoden oder Algorithmen gefunden werden die für diesen Anwendungsfall geeignet sind um ein Empfehlungssystem zu implementieren.
