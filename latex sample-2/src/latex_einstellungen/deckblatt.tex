% das Papierformat zuerst
%\documentclass[a4paper, 11pt]{article}

% deutsche Silbentrennung
%\usepackage[ngerman]{babel}

% wegen deutschen Umlauten
%\usepackage[ansinew]{inputenc}

% hier beginnt das Dokument
%\begin{document}


\thispagestyle{empty}

%\begin{figure}[t]
% \includegraphics[width=0.6\textwidth]{abb/fh_koeln_logo}
%\end{figure}

\begin{figure}[t]
 \centering
 \includegraphics[width=0.6\textwidth]{abb/logo1}
~~~~~~~~~~
 \includegraphics[width=0.3\textwidth]{abb/logo2}
\end{figure}


\begin{verbatim}


\end{verbatim}

\begin{center}
\Large{Beuth Hochschule für Technik}\\
\end{center}


\begin{center}
\Large{Fachbereich 6}
\end{center}
\begin{verbatim}




\end{verbatim}
\begin{center}
\doublespacing
\textbf{\LARGE{\titleDocument}}\\
\singlespacing
\begin{verbatim}

\end{verbatim}
\textbf{{~\subjectDocument~-~Schwerpunkt Medieninformatik}}
\end{center}
\begin{verbatim}

\end{verbatim}
\begin{center}

\end{center}
\begin{verbatim}

\end{verbatim}
\begin{center}
\textbf{zur Erlangung des akademischen Grades \\ Bachelor of Science}
\end{center}
\begin{verbatim}






\end{verbatim}
\begin{flushleft}
\begin{tabular}{llll}
\textbf{Thema:} & & Empfehlungssystem für Kompetenzen & \\
& & \\
\textbf{Autor:} & & Valentin Risch s55698@beuth-hochschule.de& \\
& & MatNr. 798906 & \\
& & \\
\textbf{Version vom:} & & \today &\\
& & \\
\textbf{1. Betreuer:} & & Prof. Dr. Johnanes Konert &\\
\textbf{2. Gutachter:} & & Prof. Dr. Y &\\
\end{tabular}
\end{flushleft}